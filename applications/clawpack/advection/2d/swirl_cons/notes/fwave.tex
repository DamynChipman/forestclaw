\documentclass{article}

\usepackage{fullpage}
\usepackage{amsmath}
\usepackage{cite}

\newcommand{\Fig}[1]{Figure \ref{fig:#1}}

\begin{document}
\section{Introduction}
\section{Numerical approaches}
The variable coefficient conservative form of the transport equation arises in many areas of
geophysical flow modeling.  For example, the transport of pollutants, volcanic ash or 
other conserved tracer component must often transported using prescibed reanalysis 
wind field data. 

In \cite{le:2002}, LeVeque describes several approaches to solving the conservative, 
variable coefficient advection equation, given by
\begin{align}
q_t + (uq)_x = 0 \\
\end{align}
In his book, LeVeque describes several approaches to solving this equation, although
there is a question about what to do in the case where the velocity field changes sign. 
In this case, there is a loss of hyperbolicity that cannot be easily addressed by the 
wave propagation algorithms implemented in ClawPack.  Here, we propose reasonable approaches
to handling this case. 

In the following, we provide details for four of the approaches described, and pay
particular attention to ways to treat the change of sign that can lead to loss of 
hyperbolicity.  The accuracy obtained from such methods is second order for velocity
fields with no sign changes.  For velocity fields with sign changes, the accuracy can be
significantly degraded, depending on the use of limiters, and whether the velocity field 
converges or diverges at the sign change.  


\subsection{Scalar approach to conservation}
In \cite{le:2002}, LeVeque describes a solution to the conservation law
\begin{align}
q_t + (uq)_x = 0 \\
\end{align}
where $u = u(x)$ is a spatially varying velocity field, and $q = q(x,t)$ is a conserved  tracer in  the flow field.  Using the analogy of boxes moving along a series of conveyor belts operating
at speeds prescribed by $u(x)$, LeVeque describes a solution involving two waves. One moves at local speeds given by each conveyor belts, and stationary waves at the interfaces between conveyor belts.  To see this, we let suppose we have two conveyor belts. The left moves at speed $u_\ell$ and the right moves at speed $u_r$.  Each conveyor belt has a density of boxes, given by $q_\ell$ and $q_r$. 
If we have $u_\ell > u_r > 0$, boxes will pile up on the right conveyor belt  as faster moving boxes from the left are transfered to the slower moving conveyor belt on the right.  If $0 < u_\ell < u_r$, the density of boxes on the right conveyor belt will drop as faster moving boxes on the right move away from boxes arriving at a slower rate from the left.  In either case, we expect a jump 
in box density both at the wave moving at speed $u_r$ along the right conveyor belt, and at the interface between the left and right conveyor belts.

We can see that single wave is not enough by considering the Riemann problem at an interface between conveyor belts.   To solve the Riemann problem, we might try to connect left and 
right states $q_\ell$ and $q_r$ by a single wave moving at some speed $s$ satisfying
\begin{equation}
u_r q_r - u_\ell q_\ell = s(q_r - q_\ell)
\end{equation}
Since $u_r \ne u_\ell$ in general, a single wave that connects the left and 
right states $q_\ell$ and $q_r$ does not exist.  Instead, we write the conservation 
law as
\begin{align*}
q_t + u_r q_x & = 0, \qquad \mbox{in the left cell} \\
q_t + u_\ell q_x & = 0, \qquad \mbox{in the right cell}
\end{align*}
and introduce two waves, one moving at speed zero, and the second moving at speed $u_r > 0$.  
We seek an intermediate state $q^*$ that can be used to connect to left and right states.
Across the wave moving at speed $u_r > 0$, we impose the condition
\begin{equation}
u_r q_r - u^* q^* = u_r(q_r - q^*)
\end{equation}
to get $u^* = u_r$.  Across the stationary wave, we impose a flux continuity condition
\begin{equation}
u^* q^* - u_\ell q_\ell = 0
\end{equation}
to get
\begin{equation*}
q^* = \frac{u_\ell q_\ell}{u_r}.
\end{equation*}
The waves moving at speeds 0 and speed $u_r > 0$ are given by
\begin{align*}
\mathcal{W}_1 & \equiv q^* - q_\ell = \frac{u_\ell q_\ell}{u_r} - q_\ell = 
-\frac{u_r - u_\ell}{u_r}q_\ell\\
\mathcal{W}_2 & \equiv q_r - q^* = q_r - \frac{u_\ell q_\ell}{u_r} = 
\frac{1}{u_r}\left(u_r q_r - u_\ell q_\ell\right)
\end{align*}
The fluctuations are then defined in terms of these waves as
\begin{align*}
\mathcal{A}^+ \Delta Q & \equiv u_r \mathcal{W}_1 = u_r q_r - u_\ell q_\ell \\
\mathcal{A}^- \Delta Q & \equiv 0
\end{align*}
The sum of the fluctuations satisfy the conservation requirement
\begin{equation*}
\mathcal{A}^+ \Delta Q + \mathcal{A}^- \Delta Q = f(q_r) - f(q_\ell)
\end{equation*}

The Riemann solver for this approach is described in \Fig{qstar}. 

\begin{figure}
\begin{center}
\begin{minipage}{0.5\textwidth}
\vspace{0.5cm}
\begin{verbatim}
# Riemann solver for "conveyor belt" approach
def rpn2(ql,qr,ul,ur):
    if ur > 0 and ul > 0:
        wave = (ur*qr - ul*ql)/ur
        amdq = 0
        apdq = ur*qr - ul*ql
        s = ur
    else if ur < 0 and ul < 0:
        wave = (ur*qr - ul*ql)/ul
        amdq = ur*qr - ul*ql
        apdq = 0
        s = ul
    else:
        uavg = (ur+ul)/2.0
        uhat = max(abs(uvg),1e-12)
        wave = (ur*qr - ul*ql)/uhat
        amdq = -ul*ql
        apdq = ur*qr
        s = uhat

    return wave,s,amdq,apdq
\end{verbatim}
\end{minipage}
\end{center}
\caption{Algorithm for cell-centered velocities based on scalar equations.}
\label{fig:qstar}
\end{figure}


\subsection{2-wave decomposition approach.}
Problem 13.11 in \cite{le:2002}
Consider the following conservation law for the conservative form of the scalar advection equation.
\begin{align}
q_t + (uq)_x &= 0 \\
u_t + (au)_x & = 0
\end{align}
for conserved quantity $q(x,t)$, and  $u = u(x,t)$ is a prescribed velocity field.  We will only 
interested in the case $a = 0$ so that $u = u(x)$ depends on space only.

We set $\mathbf q = (q,u)$ and compute the flux Jacobian $f'(\mathbf q)$
\begin{equation}
f'(\mathbf q) = \left(\begin{array}{rr} u & q \\ 0 & a\end{array}\right).
\end{equation}
The eigenvalue and eigenvector pairs for $f'(\mathbf q)$ are given by
\begin{equation}
\lambda_1 = a,  \quad  {\mathbf r}^1 = \left(\begin{array}{c} q \\ a-u\end{array}\right) \qquad \mbox{and} \qquad
\lambda_2 = u,  \quad  {\mathbf r}^2 = \left(\begin{array}{c} 1 \\ 0\end{array}\right)
\end{equation}
where we have assumed that $a < u$.  For $a \ne u$, the eigenvectors are linear independent, and so the system is strictly hyperbolic.  Furthermore, both fields satisfy $\nabla \lambda_p(\mathbf q) \cdot r^p(\mathbf q) = 0$, and so both waves are linearly degenerate.  

The first wave satisfies $\lambda_1(\mathbf q_\ell) = \lambda_1(\mathbf q_r)$ and so is only supports contact discontinuities with jumps proportional to $r^1$ and neither wave supports rarefaction
waves.  One consequence of this is that if $u_\ell < u_r$, the solution in the classic sense does not exist. 

We write the problem in quasi-linear form as 
\begin{equation}
\mathbf q_t + f'(\mathbf q) \mathbf q_x = 0
\end{equation}
or
\begin{equation}
\mathbf q_t + A(\mathbf q) \mathbf q_x = 0
\end{equation}
where $A(\mathbf q) \equiv f'(\mathbf q)$.  

Let $R = R(\mathbf q)$ be a matrix of eigen-vectors, and solve
\begin{equation}
R \alpha = \mathbf \delta,
\end{equation}
where  $\mathbf \delta = \mathbf q_r - \mathbf q_\ell$ is a general jump in
left and right states.  Solving for $\alpha$, we get 
\begin{equation}
\left[\begin{array}{c} \alpha_1 \\ \alpha_2 \end{array}\right] = \frac{1}{u-a}
\left[
\begin{array}{rr}
0 & -1 \\ u-a & q
\end{array}
\right]
\left[
\begin{array}{r}
\delta_1 \\ \delta_2
\end{array}
\right]
\end{equation}
or 
\begin{equation}
\alpha_1(\mathbf q) = \frac{-(u_r - u_\ell)}{u-a}, \qquad 
\alpha_2(\mathbf q) = q_r - q_\ell + \frac{u_r - u_l}{u-a} q
\end{equation}
To determine which values to use for $\mathbf q$, we seek values $\widehat{\mathbf q}$ 
so that 
\begin{align}
f(\mathbf q_r) - f(\mathbf q_\ell) & = A(\widehat{\mathbf q})(\mathbf q_r - \mathbf q_\ell) 
\end{align}
is satisfied.  It is straightforward to show that 
setting $\widehat{\mathbf q} = (\widehat{q},\widehat{u})$ to
\begin{equation}
\widehat{q} = \frac{q_\ell + q_r}{2} 
\qquad \mbox{and} \qquad
\widehat{u} = \frac{u_\ell + u_r}{2}
\end{equation}
we can guarantee the conservation of the method.
Using these values, we can then write
down the solution to the Riemann problem as a two-wave
solution separated by constant states

The Riemann solution to the quasi-linear problem at at interface $x=0$ can be written as 
\begin{equation}
\mathbf q(x,t) = \left\{\begin{array}{ll}
(q_\ell, u_\ell) &   x/t < a \\ 
(q^*, u^*)       &   a   \le x/t < \widehat{u} \\ 
(q_r, u_r)       &   \widehat{u} \le x/t
\end{array}
\right.
\end{equation}
where
\begin{align}
\mathbf q^* 
& = \mathbf q_\ell + \alpha_1(\widehat{\mathbf q}) {\mathbf r}^1 \\
& = \mathbf q_r - \alpha_2(\widehat{\mathbf q}) {\mathbf r}^2
\end{align}
Solving for $q^*$ and $u^*$, we get
\begin{equation}
q^*  = q_\ell - \frac{u_r - u_\ell}{\widehat{u}-a} \widehat{q}, \qquad
u^*  = u_\ell - \frac{u_r - u_\ell}{\widehat{u}-a} (a - \widehat{u}) = u_r
\end{equation}
Waves, speeds and fluctuations ae given by
\begin{align}
\mathcal W_1  \equiv \alpha_1 \mathbf r^1 & = - \frac{u_r - u_\ell}{\widehat{u}-a} 
\left(\begin{array}{c} \widehat{q} \\ a - \widehat{u}\end{array}\right) 
= (u_r - u_\ell)
\left(\begin{array}{c} \widehat{q}/\varepsilon \\ 
1 \end{array}\right) \\
\mathcal W_2  \equiv \alpha_2 \mathbf r^2 
& =  \left(q_r - q_\ell + \frac{u_r - u_\ell}{\widehat{u}-a}\widehat{q} \right)
\left(\begin{array}{c} 1 \\ 0\end{array}\right) 
= \left(\begin{array}{c} q_r - q_\ell - (u_r - u_\ell) \widehat{q}/\varepsilon \\ 
0 \end{array}\right)
\end{align}
where we have set $\varepsilon \equiv a - \widehat{u}$. 
The fluctuations are  defined as
\begin{align}
\mathcal A^- \Delta Q & \equiv \lambda_1 \alpha_1(\widehat{q}) \mathbf r^1 
= a \mathcal W_1 \\
\mathcal A^+ \Delta Q & \equiv \lambda_2 \alpha_2(\widehat{q}) \mathbf r^2 
= \widehat{u} \mathcal W_2
\end{align}
Our choice of $\widehat{\mathbf q}$ guarantees that the following conservation property is satisfied
\begin{equation}
f(\mathbf q_r) - f(\mathbf q_\ell) = \mathcal A^- \Delta Q + \mathcal A^+ \Delta Q
\end{equation}
For a spatially varying velocity field, we have $a = 0$.  In this case, $\mathcal A^- \Delta Q = 0$, $\varepsilon = -\widehat{u}$, and the waves are not defined if $u_\ell$ and $u_r$ are equal in magnitude, but opposite in sign. However, the fluctuations are still well defined.  

\begin{align}
\lim_{\varepsilon \rightarrow 0} \mathcal A^+ \Delta Q
& = \lim_{\varepsilon \rightarrow 0} -\varepsilon \mathcal W_2 \\
& = \lim_{\varepsilon \rightarrow 0} \left[-\varepsilon(q_r - q_\ell) + (u_r - u_\ell) \widehat{q}\right] \\
& = (u_r - u_\ell) \widehat{q}
\end{align}
where, for convenience, we have ignored the second component of $\mathcal W_2$. 

The Riemann solver for this approach is described in \Fig{wave_decomp}.


\begin{figure}
\begin{center}
\begin{minipage}{0.5\textwidth}
\vspace{0.25cm}
\begin{verbatim}
# Riemann solver for wave decomposition approach
def rpn2(ql,qr,ul,ur):
    qhat = (ql + qr)/2.0

    uavg = (ul + ur)/2.0      
    if uavg == 0:
        uhat = 1e-12
    else
        uhat = uavg

    s = uavg
    wave = (ur*qr - ul*ql)/uhat

    if uavg == 0:
        amdq = -ul*qhat     # amdq+apdq = (ur-ul)*qhat
        apdq = ur*qhat
    else:
        apdq = max([uhat,0])*wave
        amdq = min([uhat,0])*wave        

    return wave,s,amdq,apdq
\end{verbatim}
\end{minipage}
\end{center}
\caption{2-wave decomposition approach to handling conservative advection equation.}
\label{fig:wave_decomp}
\end{figure}

\vspace{0.25cm}

\subsection{Edge velocities}
In some sense, the simplest approach is to assume that the velocity field at cell edges is available. 
In this case, we can easily evaluate a flux at cell interfaces, and formulate the problem 
in classic flux form, where the flux is defined as
\begin{equation*}
F_{i-1/2} = \left\{\begin{array}{cc} 
u_\ell q & u_\ell, u_r < 0 \\
u_r q & u_r, u_\ell > 0
\end{array}
\right.
\end{equation*}
where

\subsection{The F-wave approach.} To correct this problem, we use the f-wave approach and
split the jump in the flux directly, avoiding the need to create waves $\mathcal W_p$.  
We decompose the jump in flux as
\begin{equation}
f(\mathbf q_r) - f(\mathbf q_\ell) = \beta_1 \mathbf r^1 + \beta_2\mathbf r^2
\end{equation}
where $\mathbf r^1$ and $\mathbf r^2$ are the eigenvectors from above, evaluated at the $\widehat{\mathbf q}$ state.   Solve this leads to 

\begin{equation}
\left[\begin{array}{c} \beta_1 \\ \beta_2 \end{array}\right] = \frac{1}{u-a}
\left[
\begin{array}{rr}
0 & -1 \\ u-a & q
\end{array}
\right]
\left[
\begin{array}{r}
\delta_1 \\ \delta_2
\end{array}
\right]
\end{equation}
or 
\begin{equation}
\beta_1(\mathbf q) = \frac{-a(u_r - u_\ell)}{\widehat{u}-a}, \qquad 
\beta_2(\mathbf q) = u_rq_r - q_rq_\ell + \frac{a(u_r - u_l)}{\widehat{u}-a} \widehat{q}
\end{equation}
F-waves are then defined as 
\begin{align}
\mathcal Z_1 & \equiv \beta_1(\mathbf q) \mathbf r^1 = - \frac{a(u_r - u_\ell)}{\widehat{u}-a} 
\left(\begin{array}{c} \widehat{q} \\ a - \widehat{u}\end{array}\right) 
= a(u_r - u_\ell) \left(\begin{array}{c} \widehat{q}/\varepsilon \\ 1 \end{array}\right) \\
\mathcal Z_2 & \equiv \beta_2(\mathbf q) \mathbf r^2 =
\left(u_r q_r - u_\ell q_\ell + \frac{a(u_r - u_\ell)}{\widehat{u}-a}\widehat{q} \right)
\left(\begin{array}{c} 1 \\ 0\end{array}\right) 
= \left(\begin{array}{c} u_r q_r - q_r q_\ell - a(u_r - u_\ell) \widehat{q}/\varepsilon \\ 
0 \end{array}\right) 
\end{align}

As in the above, we set $a = 0$.  Following the f-wave approach, we set $\mathcal A^- \Delta Q = \mathcal Z_1$ and $\mathcal A^+ \Delta Q = \mathcal Z_2$ to get
\begin{align}
\mathcal A^- \Delta Q & = 0 \\
\mathcal A^+ \Delta Q & = u_r q_r - q_r q_\ell
\end{align}
As before, the fluctuations are well-defined for all $u_\ell$, $u_r$, but now the fwaves needed by the algorithm are defined as well and are given by
\begin{align}
\mathcal Z_1 & = 0 \\
\mathcal Z_2 & = u_r q_r - q_r q_\ell
\end{align}
The f-wave approach leads to the following Clawpack Riemann solver. 

\vspace{0.25cm}

\begin{minipage}{\textwidth}
\begin{verbatim}
# Riemann solver for f-wave approach
def rpn2fwave(ql,qr,ul,ur):
    uhat = (ul + ur)/2.0
    fwave = ur*qr - ql*ul
    s = uhat
    if uhat < 0:
        amdq = wave
        apdq = 0
    else:
        amdq = 0
        apdq = wave
\end{verbatim}
\end{minipage}

\vspace{0.25cm}

\noindent
The advantage of this approach is that we don't have to make any artifical choices for the case $u_r \approx -u_\ell$.  

\section{Numerical results.}  We test the two above approaches



\bibliographystyle{plain}
\bibliography{cons}


\end{document}
