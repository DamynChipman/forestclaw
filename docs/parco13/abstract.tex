% PLEASE USE THIS FILE AS A TEMPLATE FOR THE PUBLICATION
% Check file IOS-Book-Article.tex


% preferred topic area: algorithms
% 
% at most 5 keywords
% 
% relevance
% 
% originality


\documentclass{IOS-Book-Article}     %[seceqn,secfloat,secthm]

\usepackage{mathptmx}
%\usepackage[T1]{fontenc}
%\usepackage{times}%
%
%%%%%%%%%%% Put your definitions here

\usepackage{amsmath,amssymb}
\usepackage{cite}
\usepackage{color}
\usepackage{xspace}

\newcommand{\todo}[1]{\textcolor{red}{[TODO: #1]}\xspace}
\newcommand{\sR}{\mathbb{R}}
\newcommand{\dt}{\mathrm{d}t}
\newcommand{\forestclaw}{Forestclaw\xspace}
\newcommand{\pforest}{\texttt{p4est}\xspace}

%%%%%%%%%%% End of definitions
\begin{document}
\begin{frontmatter}          % The preamble begins here.
%
%\pretitle{}
\title{\forestclaw:
        Hybrid forest-of-octrees AMR for hyperbolic conservation laws}
\runningtitle{\forestclaw}
%\subtitle{}

% Two or more authors:
%\author[A]{\fnms{} \snm{}},
%\author[B]{\fnms{} \snm{}}
%\runningauthor{}
%\address[A]{}
%\address[B]{}
%
\author[A]{\fnms{Carsten} \snm{Burstedde}%
\thanks{Corresponding author.  E-mail: \texttt{burstedde@ins.uni-bonn.de}}},
\author[B]{\fnms{Donna} \snm{Calhoun}},
\author[C]{\fnms{Kyle} \snm{Mandli}} and
\author[C]{\fnms{Andy R.} \snm{Terrel}}
\runningauthor{C.\ Burstedde et.al.}
\address[A]{Institut f\"ur Numerische Simulation, Universit\"at Bonn, Germany}
\address[B]{Boise State University, Idaho, USA}
\address[C]{Institute for Computational Engineering and Sciences,\\
The University of Texas at Austin, USA}

\begin{abstract}
%
We present a new hybrid paradigm for parallel adaptive mesh refinement (AMR)
that combines the scalability and lightweight architecture of tree-based AMR
with the computational efficiency of patch-based solvers for hyperbolic
conservation laws.  The key idea is to interpret each leaf of the AMR hierarchy
as one uniform compute patch in $\sR^d$ of dimensions $m^d$, where $m$ is
customarily between 8 and 32.  Thus, computation on each patch can be optimized
for speed, while we inherit the flexibility of adaptive meshes.  In our work we
choose to work with the \pforest AMR library since it allows to compose the mesh
from multiple mapped octrees, thus enabling the cubed sphere and other
nontrivial multiblock geometries.
%
\end{abstract}

% \begin{keyword}
% adaptive mesh refinement,
% clawpack,
% HPC,
% manycore
% \end{keyword}

\end{frontmatter}

%%%%%%%%%%% The article body starts:

\section{Introduction}

With the advent of high-throughput coprocessors, such as GPGPUs or the MIC
chip, comes the opportunity to sustain unprecedented rates of floating point
operations at comparably high integration density and low cost.  These
architectures, however, require careful structuring of the data layout and
memory access patterns to exhaust their multithreading and vectorization
capabilities.

Consequently, it is not clear a priori how to accelerate PDE solvers that use
adaptive mesh refinement, especially when working with unstructured meshes.  Of
course, it has been realized early that it helps to aggregate degrees of
freedom (DOF) at the element level, as it has been done with high-order
spectral element \cite{TufoFischer99}, finite volume
\todo{add favorite citation}, or discontinuous Galerkin
\cite{BursteddeGhattasGurnisEtAl10} methods.
%
% For example, SEs optimized for computational speed have originally been
% implemented on unstructured conforming meshes \cite{TufoFischer99} and later
% been extended to non-conforming adaptive meshes \cite{FischerKruseLoth02,
% BursteddeGhattasGurnisEtAl10}.

To enable hardware acceleration for parallel dynamic AMR, we like to build upon
the forest-of-octrees paradigm because of its low overhead and proven
scalability \cite{BursteddeWilcoxGhattas11}.  This approach identifies each
octree leaf with a mesh element.  In this work, we go
beyond the traditional high-order element and define each element to be a
dense compute patch with $m^d$ DOFs.  In fact, this approach resembles
block-structured AMR \cite{ColellaGravesKeenEtAl07}
\todo{favorite BoxLib citation}
except that the patches are not overlapping, which enables us to capitalize on
our previous experience with scalable FE solvers for PDEs
\cite{BursteddeStadlerAlisicEtAl13}.  The Clawpack software \cite{LeVeque97}
provides a popular implementation of such a patch.  It has
been designed to solve hyperbolic conservation laws
% on a uniform compute patch
and successfully used in the context of block-structured AMR \todo{AMRClaw
citation}.

In this paper we describe how we design the coupling of forest-of-octree AMR
with Clawpack at the leaf level.  We comment on challenges that arise in
enabling multiblock geometries and efficient parallelism and close with a range
of numerical examples that demonstrate the conceptual improvements in relation
to other approaches.

\section{Design principles}

The starting point of our work is defined by the \pforest algorithms for
forest-of-octrees AMR on the one hand, and the Clawpack algorithms for the
numerical solution of hyperbolic conservation laws
% on uniformly gridded domains
on the other.  Both are specialized codes with the following characteristics:
\begin{center}
\begin{tabular}{l|l|l}
& \multicolumn{1}{c|}{\pforest} & \multicolumn{1}{c}{Clawpack} \\
\hline
subject & hexahedral nonconforming mesh &  hyperbolic PDE on $[0, 1]^d$ \\
structure & forest of octrees & patch of $m^d$ DOFs \\
atomic unit & octree leaf & one DOF \\
parallelization & MPI & threads (Manyclaw variant) \\
memory access & distributed & shared \\
data type & integers & floating point values \\
language & C & Fortran 77 \\
dependencies & none & Python \\
%\hline
\end{tabular}
\end{center}
The proposed 1:1 correspondence between a leaf and a patch thus combines two
hithertwo disjoint models in a modular way:
\begin{enumerate}
\item We permit the reuse of existing, verified and performant codes.
% if the integration is done well
\item We preserve the separation of the mesh on one hand from the
discretization and solvers on the other.
\item The AMR metadata (24 bytes per leaf) is insignificant compared to
the numerical data ($m^d$ floating point values per patch).
\item The resulting parallel programming model is a hybrid (often referred
to as MPI+X).
\end{enumerate}

% yields periodicity as a special case
A particular feature of \forestclaw is that the generic handling of multiblock
geometries is inherited from \pforest, identifying each octree as a block, and
each leaf as a patch.  Each block is understood as a reference unit cube with
its own geometric mapping.  The connectivity of the blocks is static and can be
created by external hexahedral mesh generators, elimintating the need to encode
it by hand.

A main challenge is presented by the fact that the patch neighborhood is only
known to \pforest, from which it needs to be propagated to the
numerical code in \forestclaw that implements the interaction with ghost
patches.  To this end, we define an interface that allows distributed read-only
access to the sequence of blocks, the list of patches for each, and a lookup of
neighbor patches (and their relative orientation which is nontrivial between
blocks).  The exchange of ghost data and parallel partitioning is provided
transparently by \pforest.  Mesh modification directives, such as local
refinement and coarsening, are called from \forestclaw and relayed to \pforest.

% Forestclaw would support swapping out p4est or Clawpack

% AMR Metadata $\ll$ numerical data

% Experience from integration with large-scale adaptive-mesh PDE solvers.
% Design goals: Lightweight, modular, reuse.

% Mesh information is discrete (tree nodes have integer coordinates in
% $(0, 2^L($ where $L$ is the maximum allowed refinement level of the tree.

% Dimension independence

% rank as is common with parallelization using the MPI framework
% \cite{Forum94, SnirOttoHuss-LedermanEtAl96}.

\section{Parallelization}

The MPI layer is addressed from within \pforest and not exposed to the
\forestclaw code.  The order of leaves is maintained in \pforest according to a
space filling curve.  Each MPI rank has a local view on its own partition,
augmented where necessary with information about one layer of ghost patches.

\forestclaw uses iterators over all leafs, optionally restricted to a given
level.  Random access is possible and used when executing $O(1)$ time neighbor
lookups.  Looping over the patches in the order prescribed by the forest and
accessing neighbors only relative to the current patch leads to a high
percentage of cache reuse.
% since the ghost patches are likely closeby with respect to the $z$-order.

When \forestclaw accesses neighbor patches, they can be on the same or a
different block.  In the latter case, coordinate transformations are carried
out.  The structure of \forestclaw is oblivious to the fact that it only has a
local view of the distributed mesh and data which relieves the developer from
programming to the MPI interface.  Parallel neighbor exchanges are hidden
inside \pforest and called by \forestclaw at well-defined synchronization
points.  There is one such parallel data exchange per time step for a global
value of $\dt$, or one exchange per discretization level per time step if $\dt$
is chosen per-patch depending on its size (this is sometimes called
subcycling).  When using subcycling, the load balance is attained per level as
it is done in structured AMR or recent geometric adaptive multigrid schemes
\cite{SundarBirosBursteddeEtAl12}.

The threaded parallelism over the degrees of freedom of a patch is handled by
\forestclaw alone without the need to involve \pforest.

% Trees have different coordinate systems.
% neighbor block: permutation of patch points due to non-aligned coordinate
% systems

% neighbor rank: data needs to be fetched over the network first.

% $L$ is 30 for 2D and 19 for 3D, so it allows for deep hierarchies.

% fast (O(log N/P)) neighbor and parent/child lookups
% horizontal/vertical tree traversal

\todo{What is BearClaw (tree-based / S.\ Mitran)?}

\section{Patch-based numerics at the leaf level}


Patch: contains the local coordinates and copies of neighboring ghost degrees
of freedom.  The layer of ghost copies is usually 2 points deep.


\section{Numerical results}



\section*{Acknowledgements}

The authors acknowledge valuable discussion with Randy LeVeque, Marsha Berger,
and Hans-Petter Langtangen.  The leaf/patch paradigm was independently presented
by B.\ as part of a talk at the SCI Institute, Utah, in July 2011.


%%%%%%%%%%% The bibliography starts:
\bibliographystyle{unsrt}
\bibliography{ccgo}

\end{document}
